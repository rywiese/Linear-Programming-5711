\documentclass{article}
\usepackage[utf8]{inputenc}
\usepackage{amsmath,amssymb,amsthm,mathrsfs,graphicx}

\title{HW3}
\author{Ry Wiese\\wiese176@umn.edu}
\date{October 11, 2019}

\begin{document}

\maketitle

\section{Problem 1}

$A = 
\left(
    \begin{array}{cccccc}
        2 & 1 & 1 & 1 & 0 & 0\\
        3 & 1 & 2 & 0 & 1 & 0\\
        1 & 2 & 4 & 0 & 0 & 1\\
    \end{array}
\right),~
b = 
\left(
    \begin{array}{c}
        240\\
        150\\
        180\\
    \end{array}
\right),~ 
c = 
\left(
    \begin{array}{c}
        500\\
        250\\
        600\\
        0\\
        0\\
        0\\
    \end{array}
\right)$

$\mathbf{Step~1}$:\\
Basis: $B = \{4, 5, 6\}$\\
BFS: $\mathbf{x}_B = A_B^{-1}\mathbf{b} = 
\left(
    \begin{array}{c}
        240\\
        150\\
        180\\
    \end{array}
\right)$\\
Objective value: $\mathbf{c}^T \mathbf{x} = 0$\\
Reduced costs: $\mathbf{\bar{c}}_N = \mathbf{c}_N - (A_B^{-1} A_N)^T \mathbf{c}_B = 
\left(
    \begin{array}{c}
        -500\\
        -250\\
        -600\\
    \end{array}
\right)$\\
Variable to enter: $x_1$, since $\mathbf{\bar{c}}_{N_1} = -500$ is the first negative value in $\mathbf{\bar{c}}_N$\\
Variable to exit: $x_5$, since $\mathbf{d}_B = \left(
    \begin{array}{c}
        -2\\
        -3\\
        -1\\
    \end{array}
\right),~
\mathbf{\theta}_B = \left(
    \begin{array}{c}
        120\\
        50\\
        180\\
    \end{array}
\right)$ with $\theta_{B_5} = 50$\\
We change to basis $\{1,4,6\}$ since $\mathbf{\bar{c}}$ is not semipositive.\\
\\
\\
$\mathbf{Step~2}$:\\
Basis: $B = \{1, 4, 6\}$\\
BFS: $\mathbf{x}_B = A_B^{-1}\mathbf{b} = 
\left(
    \begin{array}{c}
        50\\
        140\\
        130\\
    \end{array}
\right)$\\
Objective value: $\mathbf{c}^T \mathbf{x} = -25,000$\\
Reduced costs: $\mathbf{\bar{c}}_N = \mathbf{c}_N - (A_B^{-1} A_N)^T \mathbf{c}_B = 
\left(
    \begin{array}{c}
        -83\\
        -267\\
        -167\\
    \end{array}
\right)$\\
Variable to enter: $x_2$, since $\mathbf{\bar{c}}_{N_2} = -83$ is the first negative value in $\mathbf{\bar{c}}_N$\\
Variable to exit: $x_6$, since $\mathbf{d}_B = \left(
    \begin{array}{c}
        -.3\\
        -.3\\
        -1.7\\
    \end{array}
\right),~
\mathbf{\theta}_B = \left(
    \begin{array}{c}
        150\\
        420\\
        78\\
    \end{array}
\right)$ with $\theta_{B_6} = 78$\\
We change to basis $\{1,2,4\}$ since $\mathbf{\bar{c}}$ is not semipositive.\\
\\
\\
$\mathbf{Step~3}$:\\
Basis: $B = \{1, 2, 4\}$\\
BFS: $\mathbf{x}_B = A_B^{-1}\mathbf{b} = 
\left(
    \begin{array}{c}
        24\\
        78\\
        114\\
    \end{array}
\right)$\\
Objective value: $\mathbf{c}^T \mathbf{x} = -31,500$\\
Reduced costs: $\mathbf{\bar{c}}_N = \mathbf{c}_N - (A_B^{-1} A_N)^T \mathbf{c}_B = 
\left(
    \begin{array}{c}
        -100\\
        -150\\
        -50\\
    \end{array}
\right)$\\
Variable to enter: $x_3$, since $\mathbf{\bar{c}}_{N_3} = -100$ is the first negative value in $\mathbf{\bar{c}}_N$\\
Variable to exit: $x_2$, since $\mathbf{d}_B = \left(
    \begin{array}{c}
        0\\
        -2\\
        1\\
    \end{array}
\right),~
\mathbf{\theta}_B = \left(
    \begin{array}{c}
        -\\
        39\\
        -\\
    \end{array}
\right)$ with $\theta_{B_2} = 39$\\
We change to basis $\{1,3,4\}$ since $\mathbf{\bar{c}}$ is not semipositive.\\
\\
\\
$\mathbf{Step~4}$:\\
Basis: $B = \{1, 3, 4\}$\\
BFS: $\mathbf{x}_B = A_B^{-1}\mathbf{b} = 
\left(
    \begin{array}{c}
        24\\
        39\\
        153\\
    \end{array}
\right)$\\
Objective value: $\mathbf{c}^T \mathbf{x} = -35,400$\\
Reduced costs: $\mathbf{\bar{c}}_N = \mathbf{c}_N - (A_B^{-1} A_N)^T \mathbf{c}_B = 
\left(
    \begin{array}{c}
        50\\
        140\\
        80\\
    \end{array}
\right)$\\
The current basis, $B = \{1, 3, 4\}$ is optimal because $\mathbf{\bar{c}} \ge \mathbf{0}$. \\
\\
Thus the optimal solution is $x_1 = 24, x_2 = 0,$ and $x_3 = 39$ for an optimal value of 35,400 (since the original problem was maximization).

\section{Problem 2}

$\mathbf{Step~1}$:\\
Basis: $B = \{5, 6\}$\\
BFS: $\mathbf{x}_B = A_B^{-1}\mathbf{b} = 
\left(
    \begin{array}{c}
        0\\
        0\\
    \end{array}
\right)$\\
Objective value: $\mathbf{c}^T \mathbf{x} = 0$\\
Reduced costs: $\mathbf{\bar{c}}_N = \mathbf{c}_N - (A_B^{-1} A_N)^T \mathbf{c}_B = 
\left(
    \begin{array}{c}
        -2\\
        -3\\
        1\\
        12\\
    \end{array}
\right)$\\
Variable to enter: $x_1$, since $\mathbf{\bar{c}}_{N_1} = -2$ is the first negative value in $\mathbf{\bar{c}}_N$\\
Variable to exit: $x_6$, since $\mathbf{d}_B = \left(
    \begin{array}{c}
        2\\
        -.3\\
    \end{array}
\right),~
\mathbf{\theta}_B = \left(
    \begin{array}{c}
        -\\
        0\\
    \end{array}
\right)$ with $\theta_{B_6} = 0$\\
We change to basis $\{1,5\}$ since $\mathbf{\bar{c}}$ is not semipositive.\\
\\
\\
$\mathbf{Step~2}$:\\
Basis: $B = \{1,5\}$\\
BFS: $\mathbf{x}_B = A_B^{-1}\mathbf{b} = 
\left(
    \begin{array}{c}
        0\\
        0\\
    \end{array}
\right)$\\
Objective value: $\mathbf{c}^T \mathbf{x} = 0$\\
Reduced costs: $\mathbf{\bar{c}}_N = \mathbf{c}_N - (A_B^{-1} A_N)^T \mathbf{c}_B = 
\left(
    \begin{array}{c}
        3\\
        -1\\
        0\\
        6\\
    \end{array}
\right)$\\
Variable to enter: $x_3$, since $\mathbf{\bar{c}}_{N_3} = -1$ is the first negative value in $\mathbf{\bar{c}}_N$\\
$\mathbf{d}_B = \left(
    \begin{array}{c}
        1\\
        1\\
    \end{array}
\right) \ge \mathbf{0}$, therefore the problem is unbounded.

\section{Problem 3}

Let $B \subseteq \{1,..,n+m\}$ be the basis returned by $\Phi$. Suppose $B$ can be partitioned into $\{P,Q\}$ where $P \subseteq \{1,..,n\}$ (indices of non-artificial variables) and $Q \subseteq \{n+1,..,n+m\}$ (indices of artificial variables). Let $A' = (A | I_m)$ be the matrix used in $\Phi$. Since the optimal value $\sum_{j=1}^m x_{n+j} = 0$ and $\mathbf{x} = 0$, $\left( \begin{array}{c} x_n+1\\...\\x_n+m \end{array} \right) = \mathbf{0}$. Then, if $\mathbf{x}$ is a BFS of $\Phi$, $A' \mathbf{x} = A \left( \begin{array}{c} x_1\\...\\x_n \end{array} \right) + I_m \left( \begin{array}{c} x_n+1\\...\\x_n+m \end{array} \right) = A \left( \begin{array}{c} x_1\\...\\x_n \end{array} \right)$. Thus the basis vectors defined in $Q$ do not contribute anything to the BFS, as they have scalar coefficients of 0. This means we can replace the indices in $Q$ with $\textit{any}$ indices in $B$ that do not create a linear dependence. 


\section{Problem 4}

My algorithm is contained in the Simplex function, which uses Simplex$\_$Helper as a helper function.You should be able to use Simplex without needing to touch Simplex$\_$Helper at all. I have also created a function Test(m, n) which randomly generates $A$, $\mathbf{b}$, and $\mathbf{c}$, prints the output of the actual solution using cvx, and then prints my solution using my algorithm.

\end{document}
